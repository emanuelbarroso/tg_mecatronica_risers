%TCIDATA{LaTeXparent=0,0,relatorio.tex}
 


\chapter{Introdu��o}

\label{CapIntro}

% Resumo opcional. Comentar se n�o usar.
\resumodocapitulo{Resumo opcional}


\section{Contextualiza��o}

Contextualizar.

Conforme \cite{6913799}, vide a Tabela \ref{tab:Descrever-tabela}.
Assim sendo, observe a Figura \ref{fig:Descrever-figura.}. 
\begin{table}[h]
\begin{centering}
\begin{tabular}{|c|c|c|c|c|}
\hline 
 &  &  &  & \tabularnewline
\hline 
\hline 
 &  &  &  & \tabularnewline
\hline 
 &  &  &  & \tabularnewline
\hline 
 &  &  &  & \tabularnewline
\hline 
 &  &  &  & \tabularnewline
\hline 
\end{tabular}
\par\end{centering}

\caption{\label{tab:Descrever-tabela}Descrever tabela.}


\end{table}


\begin{figure}[h]
\begin{centering}
\includegraphics[width=0.4\columnwidth]{figs/capa_fundo}
\par\end{centering}

\caption{\label{fig:Descrever-figura.}Descrever figura.}


\end{figure}



\section{Defini��o do problema}

Definir problema.


\section{Objetivos do projeto}

Objetivos.


\section{Resultados obtidos}

Resultados.


\section{Apresenta��o do manuscrito}

Apresentar.
