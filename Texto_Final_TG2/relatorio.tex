\documentclass[a4paper,oneside,english,brazil,11pt,a4paper,openright,titlepage,usenames,dvipsnames]{book}
\usepackage[T1]{fontenc}
\usepackage[utf8]{inputenc}
\usepackage{lmodern}
\setcounter{secnumdepth}{3}
\setcounter{tocdepth}{3}
\usepackage{array}
\usepackage{verbatim}
\usepackage{calc}
\usepackage{graphicx}
\usepackage{textcomp}
\usepackage{listings}

%PDF table of contents
\usepackage{hyperref}
\hypersetup{pdftex,colorlinks=true,allcolors=blue}
\usepackage{hypcap}
% -------------------

\makeatletter

%%%%%%%%%%%%%%%%%%%%%%%%%%%%%% LyX specific LaTeX commands.
\pdfpageheight\paperheight
\pdfpagewidth\paperwidth

%% Because html converters don't know tabularnewline
\providecommand{\tabularnewline}{\\}

%%%%%%%%%%%%%%%%%%%%%%%%%%%%%% User specified LaTeX commands.
% Classe alternativa, apropriada para impressão frente-verso. Inclui páginas em branco
% de forma que capítulos sempre tenham início na página à direita:
% \documentclass[11pt,a4paper,openright,titlepage]{book}

% Pacotes
\usepackage[brazilian]{babel}
\usepackage{epsfig}
\usepackage{subfigure}
\usepackage{amsfonts}
\usepackage{amsmath}
\usepackage[thmmarks,amsmath]{ntheorem}%\usepackage{amsthm}
\usepackage{boxedminipage}
\usepackage{geometry}
\usepackage{theorem}
\usepackage{fancybox}
\usepackage{fancyhdr}
\usepackage{ifthen}
\usepackage{url}
\usepackage{afterpage}
\usepackage{color}
\usepackage{colortbl}
\usepackage{rotating}
\usepackage{makeidx}
\usepackage{indentfirst}
% Pacotes para adiçao de figuras do inkscape
\usepackage{graphicx}
\usepackage{import}
%\usepackage[usenames,dvipsnames]{color}

\usepackage{tikz}
\usetikzlibrary{positioning}
\usepackage{textcomp}
\usepackage{tabularx}
\usetikzlibrary{shapes,arrows}

% Escolher um dos seguintes formatos:
%\usepackage{ft1unb} % segue padrão de fonte Times
\usepackage{ft2unb} % segue padrão de fontes do Latex

\makeindex
% Added by lyx2lyx
\usepackage{amssymb}

\makeatother

\usepackage{babel}

\usepackage{beramono}
\usepackage{listingsutf8}
%\usepackage[usenames,dvipsnames]{xcolor}

%%
%% Julia definition (c) 2014 Jubobs
%%
\lstdefinelanguage{Julia}%
  {morekeywords={abstract,break,case,catch,const,continue,do,else,elseif,%
      end,export,false,for,function,immutable,import,importall,if,in,%
      macro,module,otherwise,quote,return,switch,true,try,type,typealias,%
      using,while},%
   sensitive=true,%
   alsoother={\$},%
   morecomment=[l]\#,%
   morecomment=[n]{\#=}{=\#},%
   morestring=[s]{"}{"},%
   morestring=[m]{'}{'},%
}[keywords,comments,strings]%

\lstset{%
    language         = Julia,
    basicstyle       = \ttfamily,
    keywordstyle     = \bfseries\color{blue},
    stringstyle      = \color{magenta},
    commentstyle     = \color{ForestGreen},
    showstringspaces = false
}

\lstset{
	language = Python,
	basicstyle=\ttfamily\small,
    numberstyle=\footnotesize,
    numbers=left,
    backgroundcolor=\color{gray!10},
    frame=single,
    tabsize=2,
    rulecolor=\color{black!30},
    title=\lstname,
    escapeinside={\%*}{*)},
    breaklines=true,
    breakatwhitespace=true,
    framextopmargin=2pt,
    framexbottommargin=2pt,
    extendedchars=true,
    inputencoding=utf8,
    literate={á}{{\'a}}1 {ã}{{\~a}}1 {é}{{\'e}}1 {ç}{{\c{c}}}1 {â}{{\^a}}1 {õ}{{\~o}}1 {ú}{{\'u}}1 {ó}{{\'o}}1 {í}{{\'i}}1 {Í}{{\'I}}1 
}


\begin{document}
\setcounter{secnumdepth}{3}
\setcounter{tocdepth}{2}
\pagestyle{empty}

\grau{Engenheiro de Controle e Automação}

\tipodemonografia{TRABALHO DE GRADUAÇÃO}

\begin{comment}
Título
\end{comment}


\titulolinhai{PROJETO E IMPLEMENTAÇÃO EM CONTROLADOR}

\titulolinhaii{INDUSTRIAL PARA POSICIONAMENTO DE RISERS}

\titulolinhaiii{COM VALIDAÇÃO EXPERIMENTAL}

\titulolinhaiv{}

%Autores. Basta retirar o texto totalmente caso não haja um determinado
%autor.


\autori{Ataias Pereira Reis}

\autorii{Emanuel Pereira Barroso Neto}

\autoriii{}

%Membros da banca. Basta retirar o texto totalmente caso não haja um
%determinado membro da banca.


\membrodabancai{Prof. Eugênio Libório Feitosa Fortaleza, ENM/UnB}

\membrodabancaifuncao{Orientador}

\membrodabancaii{Prof. Eduardo Stockler Tognetti, ENE/UnB}

\membrodabancaiifuncao{Co-orientador}

\membrodabancaiii{Prof. Guilherme Caribé de Carvalho, ENM/UnB}

\membrodabancaiiifuncao{}

\membrodabancaiv{Prof. Geovany Araújo Borges, ENE/UnB}

\membrodabancaivfuncao{}

\membrodabancav{}

\membrodabancavfuncao{}

\begin{comment}
Data de defesa: mês e ano
\end{comment}


\mes{julho} 
\ano{2016}

%Comandos para criar a capa e a página de assinaturas

\capaprincipal 
\capaassinaturas

%Ficha Catalográfica


\noindent \textbf{FICHA CATALOGRÁFICA}

\noindent %
\framebox{\begin{minipage}[t]{1\columnwidth}%
REIS, ATAIAS PEREIRA; NETO, EMANUEL PEREIRA BARROSO;

Projeto e Implementação em Controlador Industrial para Posicionamento de Risers com Validação Experimental ,

\medskip{}


{[}Distrito Federal{]} 2016.

\medskip{}


x, 65p., 297 mm (FT/UnB, Engenheiro, Controle e Automação, 2016).
Trabalho de Graduação \textendash{} Universidade de Brasília.Faculdade
de Tecnologia.

\medskip{}


1. Controle em Malha Fechada\hfill{} 2. Controlador Lógico Programável\hfill{}

3. Sistemas Offshore

\medskip{}


I. Mecatrônica/FT/UnB\hfill{} %II. Título (Série)\hfill{}

%
\end{minipage}}

\noindent \medskip{}


\noindent \textbf{REFERÊNCIA BIBLIOGRÁFICA}

REIS, A. P., NETO, E. P. B. (2016). Projeto e Implementação em Controlador Industrial para Posicionamento de Risers com Validação Experimental. Trabalho de Graduação em Engenharia de Controle e Automação, Publicação FT.TG-$n^{\circ}0XX$, Faculdade de Tecnologia, Universidade
de Brasília, Brasília, DF, 64p.

\noindent \bigskip{}


\noindent \textbf{CESSÃO DE DIREITOS}

\noindent AUTORES: Ataias Pereira Reis e Emanuel Pereira Barroso Neto

TÍTULO DO TRABALHO DE GRADUAÇÃO: Projeto e Implementação em Controlador Industrial para Posicionamento de Risers com Validação Experimental.
\noindent \medskip{}


\noindent GRAU: Engenheiro\hfill{}ANO: 2016\hfill{}

\noindent \medskip{}


É concedida à Universidade de Brasília permissão para reproduzir cópias
deste Trabalho de Graduação e para emprestar ou vender tais cópias
somente para propósitos acadêmicos e científicos. Os autores reservam
outros direitos de publicação e nenhuma parte desse Trabalho de Graduação
pode ser reproduzida sem autorização por escrito dos mesmos.

\noindent \bigskip{}

\noindent \begin{tabular}{ll}
	\rule[0.5ex]{0.5\columnwidth}{1pt} & \rule[0.5ex]{0.5\columnwidth}{1pt}\\
	Ataias Pereira Reis & Emanuel Pereira Barroso Neto\\
	72926-048 Águas Lindas \textendash{} GO \textendash{} Brasil & 72130-450 Taguatinga \textendash{} DF \textendash{} Brasil
\end{tabular}

%Dedicatória


\frontmatter

%Texto de dedicatória do primeiro autor.


\dedicatoriaautori{A Emanuel B., meu avô, fonte de inspiração eterna}%
\begin{comment}
Texto de dedicatória do segundo autor. Caso não tenha um segundo autor,
este texto não será mostrado 
\end{comment}


\dedicatoriaautorii{Dedico à minha família, que me apoiou ao longo desta jornada}%
%Texto de dedicatória do terceiro autor. Caso não tenha um terceiro
%autor, este texto não será mostrado 

\dedicatoriaautoriii{Dedicatória do autor 3}

\begin{comment}
Comando para criar a página de dedicatória
\end{comment}


\dedicatoria 

%Agradecimentos
%Texto de agradecimentos do primeiro autor.


\agradecimentosautori{Eu agradeço a Deus, aos professores Eugênio Fortaleza e Eduardo Tognetti, orientadores deste trabalho, e a todos que realizaram trabalhos prévios que foram utilizados aqui. Agradeço também ao solícito Rédytton Brenner que nos ajudou muito no início.
}

\begin{comment}
Texto de agradecimentos do segundo autor. Caso não tenha um segundo
autor, este texto não será mostrado.
\end{comment}


\agradecimentosautorii{Agradeço a Deus, aos professores Eugênio Fortaleza e Eduardo Tognetti, orientadores deste trabalho, sempre disponíveis quando mais precisamos; ao grande amigo Rédytton Brenner, que utilizou a bancada antes de nós e nos deu várias dicas úteis; a todos os meus amigos e familiares, que me deram forças nessa jornada; por fim, agradeço à minha dupla, Ataias, pela paciência, cordialidade e principalmente pela ajuda.

}

\begin{comment}
Texto de agradecimentos do terceiro autor. Caso não tenha um terceiro
autor, este texto não será mostrado.
\end{comment}


\agradecimentosautoriii{A inclusão desta seção de agradecimentos
é opcional e fica à critério do(s) autor(es), que caso deseje(em)
inclui-la deverá(ao) utilizar este espaço, seguindo está formatação.

}

\begin{comment}
Comando para criar a página de agradecimentos
\end{comment}


\agradecimentos

\begin{comment}
Inclusão de Macros
\end{comment}


\selectlanguage{english}%


\global\long\def\mymatrix#1{\mathbf{#1}}


\global\long\def\vect#1{\vec{#1}}


\global\long\def\dualnum#1{\underline{#1}}




\global\long\def\dotprod#1{\langle#1\rangle}
\global\long\def\norm#1{\parallel#1\parallel}




\global\long\def\quat#1{\mathbf{#1}}
\global\long\def\dq#1{\underline{\mathbf{#1}}}
 

\global\long\def\dualvec#1{\underline{\vec{#1}}}




\global\long\def\hamiltonpos#1{\overset{+}{\mymatrix H}(#1)}


\global\long\def\hamiltonneg#1{\overset{-}{\mymatrix H}(#1)}




\global\long\def\getdual#1{Du(#1)}


\global\long\def\getprimary#1{Pr(#1)}




\global\long\def\tplus{\overset{+}{\underline{\mathbf{t}}}}




\global\long\def\frame#1{Q^{#1}}


\global\long\def\orientation#1#2{O_{#2}^{#1}}


\global\long\def\pose#1#2{P_{#2}^{#1}}


\global\long\def\gettranslacao#1{\mathbf{T}(#1)}


\global\long\def\getorientacao#1{\mathbf{O}(#1)}




\global\long\def\qunitariomudancadequadro#1#2#3{\quat{#1}_{#2\rightarrow#3}}




\global\long\def\translation#1#2{\mathbf{t}_{#1\rightarrow#2}}




\global\long\def\mpqd#1#2#3#4{\dq{#1}_{#2\rightarrow#3}^{#4}}


\global\long\def\transicao#1#2#3#4{#1_{#2\rightarrow#3}^{#4}}


\global\long\def\mcd{MCD}


\global\long\def\mci{MCI}


\global\long\def\mcdif{MCDif}


\global\long\def\mcdifinv{MCDifInv}


\global\long\def\dh{DH}


\global\long\def\jacob{J}




\global\long\def\rhino{Rhino\: XR4}


\global\long\def\home{home}


\global\long\def\ambiente{\text{Linux/Xenomai}}


\global\long\def\pio{PIO-821}


\global\long\def\ciint{8259}


\global\long\def\piacionamento{P.I.-Acionamento}


\global\long\def\piaquisicao{P.I.-Aquisi\text{{ç}}\tilde{a}o}


\global\long\def\placai{\text{{P.I.}}}


\global\long\def\schu{\text{{Schunk}}}
\selectlanguage{brazil}%



\resumo{resumo}{
A extração de petróleo em águas profundas requer operações bastante complexas. Em especial, a operação de entrada reentrada ocorre quando um \textit{riser} é conectado de sua plataforma a um poço de petróleo. O deslocamento do \textit{riser} até o poço deve ser feito de forma rápida e precisa, mas, atualmente, o controle de posição é manual, o que pode ser ineficiente. O presente trabalho busca validar técnicas de controle por meio de experimentos em uma planta de laboratório representativa de um \textit{riser} por meio de controle em malha aberta e fechada para operar o deslocamento do \textit{riser} de forma automática, com resultados rápidos e precisos, além de oferecer resistência a perturbações causadas pelo oceano, para que a ponta do \textit{riser} seja corretamente conectada ao poço.

\medskip{}


Palavras Chave: \textit{riser}, controle, deslocamento

}\vspace*{2cm}


\resumo{Abstract}{
Deep sea petroleum exploration requires very complex operations. Particularly, the re-entry operation occurs when a riser is connected from the platform to a wellhead. The riser's displacement must be done quickly and precisely but, presently, the position control is manual, which may be inefficient. The present paper seeks to validate control techniques with experiments in a \textit{riser} representative lab plant using open and closed loop control to operate the riser's displacement automatically, with fast and precise results, in addition to offering resistance against disturbances caused by the ocean, so that riser's tip is correctly connected to the wellhead.

\medskip{}


Keywords: riser, control, displacement

}

\begin{comment}
Listas de conteúdo, figuras e tabelas
\end{comment}


\sumario 
\listadefiguras 
\listadetabelas

\begin{comment}
Lista de Símbolos
\end{comment}


%TCIDATA{LaTeXparent=0,0,these.tex}


%\chapter*{\setfontarial\mdseries LISTA DE SÍMBOLOS} % se usar ft1unb.sty, descomente esta linha



\chapter*{LISTA DE SÍMBOLOS}

% se usar ft2unb.sty, descomente esta linha


%TODO adicionar constantes ao longo da escrita do relatório
\subsection*{Símbolos Latinos}

\begin{tabular}{p{0.1\textwidth}p{0.63\textwidth}>{\PreserveBacklash\raggedleft}p{0.15\textwidth}}
$v$  & Velocidade linear  & {[}m/s{]}\tabularnewline
$r$ & Referência de posição & {[}m{]}\tabularnewline
$\mathbf{W}$ & Letra em negrito representa matriz & \tabularnewline
\end{tabular}

%TODO Adicionar mu, tau, tau', Upsilon, etc.
\subsection*{Símbolos Gregos}

\begin{tabular}{p{0.1\textwidth}p{0.63\textwidth}>{\PreserveBacklash\raggedleft}p{0.15\textwidth}}
$\Upsilon$ & Deslocamento horizontal & {[}m{]}\tabularnewline
$\epsilon$ & Atraso & {[}s{]}\tabularnewline
\end{tabular}


%TODO revisar e completar Subscritos e Sobrescritos - deve-se adicionar o sobrescrito '~'
\subsection*{Subscritos}

\begin{tabular}{p{0.1\textwidth}p{0.8\textwidth}}
$M$ & Sistema Modal\tabularnewline
$R$ & Sistema Reduzido\tabularnewline
$D$ & Sistema Reduzido considerando Atraso\tabularnewline
\end{tabular}


\subsection*{Sobrescritos}

\begin{tabular}{p{0.1\textwidth}p{0.8\textwidth}}
$\cdot$  & Variação temporal \tabularnewline
$-$  & Valor médio \tabularnewline
\textasciicircum & Estimação \tabularnewline
\end{tabular}


\subsection*{Siglas}

\begin{tabular}{p{0.2\textwidth}p{0.8\textwidth}}
PCI  & \textit{Peripheral Component Interconnect}\tabularnewline
CPU & Unidade Central de Processamento - \textit{Central Processing Unit} \tabularnewline
CAN & \textit{Control Area Networking} \tabularnewline
CIP & Protocolo Industrial Comum - \textit{Common Industrial Protocol} \tabularnewline
CLP & Controlador Lógico Programável\tabularnewline
PWM &  Modulação por Largura de Pulso - \textit{Pulse Width Modulation}\tabularnewline
SI & Sistema Internacional de Unidades \tabularnewline
EDS & \textit{Electronic Data Sheet}
\tabularnewline
RSLogix & \textit{Rockwell Software Logix 5000}
\tabularnewline
RSLogix5000 & \textit{Rockwell Software Logix 5000}
\tabularnewline
OLE & \textit{Object Linking and Embedding}\tabularnewline
OPC & \textit{OLE for Process Control}\tabularnewline
SISO & \textit{Singular Input, Singular Output}\tabularnewline
\end{tabular}


\begin{comment}
Corpo Principal
\end{comment}


\mainmatter 
\setcounter{page}{1} 
\pagenumbering{arabic} 
\pagestyle{plain}

%Introdução
%TCIDATA{LaTeXparent=0,0,relatorio.tex}
 


\chapter{Introdução}

\label{CapIntro}

% Resumo opcional. Comentar se não usar.
\resumodocapitulo{A motivação do presente trabalho aparece na área petrolífera, nas operações de reentrada de risers e o objetivo é a validação de um sistema de controle em malha fechada em escala laboratorial como uma alternativa ao modelo manual empregado atualmente.}


\section{Contextualização}

O petróleo tem importância econômica global. Uma das formas de extração do mesmo é aquela feita em águas profundas, a qual o Brasil tem feito importantes avanços desde a descoberta do Pré-Sal em 2006. Em abril de 2015, chegou-se à produção de mais de 800 mil barris por dia no pré-sal, com campos situados em águas profundas e ultraprofundas \cite{preSal}. Os desafios nesta área da engenharia são enormes, pois as operações são muito complexas.

Na Figura \ref{riser}, observa-se uma das operações necessárias para a extração no Pré-Sal, especificamente a operação de reentrada. Nesta operação, um \textit{riser} deve ser conectado da plataforma até o poço de petróleo no leito oceânico. O comprimento do \textit{riser} chega a 2km.

Devido à complexidade inerente das diversas operações \textit{offshore}, propulsores e sensores de localização e orientação - GPS, giroscópios, câmeras, etc - são requisitos essenciais para se poder posicionar a embarcação e os \textit{risers} \cite{redytton}.

\begin{figure}[ht!]
\centering
  \includegraphics[width=5cm]{figs/introducao/riser}
  \caption{Operação de reentrada \cite{eugenioASME2012}\label{riser}}
\end{figure}


\section{Objetivos do projeto}

O foco deste trabalho está na operação de reentrada, conforme apresentada na Figura \ref{riser}. Atualmente, é uma operação feita manualmente por um operador na plataforma que observa remotamente as imagens capturas por \textit{ROV}s (\textit{Remotely Operated Vehicles}) da região do \textit{riser} próxima ao poço e controla a plataforma através de um \textit{joystick} com o auxílio do sistema de posicionamento dinâmico. O custo envolvido na operação é enorme e os riscos para o equipamento também, já que os próprios \textit{ROVs} e o \textit{riser} ficam sujeitos às perturbações das ondas, correntezas e variações ambientais no fundo do mar. A Figura \ref{posicionamentoAtual} apresenta o esquema que acabou de ser descrito.

\begin{figure}[ht!]
\centering
  \includegraphics[width=7cm]{figs/introducao/posicionamentoAtual}
  \caption{Método atual para reconexão no poço \cite{redytton} \label{posicionamentoAtual}}
\end{figure}

O objetivo deste trabalho é apresentar uma forma mais eficiente de realizar a operação de reentrada, economizando tempo e recursos, evitando riscos para pessoal e equipamento. Rédytton \cite{redytton} validou o controle em malha aberta. Neste presente trabalho, deseja-se revalidar a malha aberta, com pequenas mudanças na excursão total do carrinho em comparação com o trabalho de Rédytton, assim como realizar as configurações necessárias para fechar a malha.

\begin{figure}[ht!]
\centering
  \includegraphics[width=7cm]{figs/introducao/posicionamentoProposto}
  \caption{Método proposto para reconexão no poço \cite{redytton} \label{posicionamentoProposto}}
\end{figure}

Para fechar a malha, necessita-se de um sensor que realimente os dados de posição, o qual, na presente bancada, é uma câmera industrial. Passos necessários incluem: \begin{itemize}
	\item Alimentação da câmera;
	\item Atualização do Firmware da Câmera
	\item Conexão na rede Ethernet/IP da câmera e CLP;
	\item Configuração da câmera no software RSLogix;
	\item Alimentação da bancada com fontes dedicadas de 24V\footnote{Anteriormente, duas fontes de tensão de saída ajustável eram utilizadas, mas foram trocadas por fontes de saída de tensão única.}
\end{itemize}



%Fundamentos
%TCIDATA{LaTeXparent=0,0,relatorio.tex}
\chapter{Fundamentos\label{chap:FundamentacaoMatematica}}

% Resumo opcional. Comentar se não usar.
\resumodocapitulo{Este capítulo apresenta equações básicas do sistema que se deseja validar, assim como informações sobre a bancada laboratorial e também sobre a programação do CLP.}

\section{Equações Governantes}

Estruturas submarinas tais como os \textit{risers} são esbeltas e tem um alto módulo de cisalhamento. Portanto, a simplificação de Euler-Bernoulli para vigas é utilizada para propósitos de modelagem. O deslocamento de interesse é o horizontal e o \textit{riser} está sob a ação de forças hidrodinâmicas externas e de tração. A equação diferencial parcial para a variável deslocamento, $\Upsilon$, é dada por \begin{align}
	m_s \frac{\partial^2 \Upsilon}{\partial t^2} &= -E J	\frac{\partial^4 \Upsilon}{\partial z^4} + \frac{\partial}{\partial z}\left(T(z) \frac{\partial \Upsilon}{\partial z}\right) + F_n(z,t),
\end{align} na qual $m_s$ é a densidade linear do tubo, $E$ é o módulo de Young e $J$ é o segundo momento de inércia do \textit{riser}. $T(z)$ descreve as forças de tração ao longo do comprimento do \textit{riser}. $F_n(z,t)$ é a força resultante externa \cite{fabricioIFAC}.

As únicas forças externas atuando no \textit{riser} são hidrodinâmicas, exceto nas extremidades do topo e do fundo, nas quais forças de reação seguem condições de contorno. A equação de Morison descreve a força externa resultante: \begin{align}
	F_n(z,t) &= -m_f \frac{\partial^2 \Upsilon}{\partial t^2} - \mu\left|\frac{\partial \Upsilon}{\partial t}\right|\frac{\partial \Upsilon}{\partial t},
\end{align} na qual $m_f$ é a massa do fluido adicionado e $\mu$ é o coeficiente de arrasto. Seja $m = m_s + m_f$ 

\section{Controle}
\paragraph{} Técnicas de controle simples devem ser introduzidas de forma a se compreender o objetivo deste trabalho, que é do posicionamento do \textit{riser} por meio de controle em malha fechada. Primeiramente, apresenta-se o controle em malha aberta, cujo diagrama pode ser observado na Figura \ref{mabertatikz}. A variável $r = r(t)$ é a referência do sistema. Neste tipo de controle, a saída não é realimentada na entrada. Desta forma, este tipo de controle não requer sensores, pois somente dá uma referência de entrada que a planta deve seguir. Caso haja erros para seguir a trajetória, eles não poderão ser compensados e é um controle mais recomendado quando o sistema é preciso e há pouca ou nenhuma perturbação. No entanto, este não é o caso do \textit{riser}, pois o movimento das águas no leito oceânico perturba o tubo, causando erros na posição final desejada.

\tikzstyle{block} = [draw, fill=blue!20, rectangle, 
minimum height=3em, minimum width=6em]
\tikzstyle{sum} = [draw, fill=blue!20, circle, node distance=1cm]
\tikzstyle{input} = [coordinate]
\tikzstyle{output} = [coordinate]
\tikzstyle{pinstyle} = [pin edge={to-,thin,black}]

\begin{figure}[!ht]
	\centering
	% The block diagram code is probably more verbose than necessary
	\begin{tikzpicture}[auto, node distance=2cm,>=latex']
	% We start by placing the blocks
	\node [input, name=input] {};
	\node [block, right of=input] (controller) {Controlador};
	\node [block, right of=controller, pin={[pinstyle]above:Perturbações},
	node distance=3cm] (system) {Planta};
	% We draw an edge between the controller and system block to 
	% calculate the coordinate u. We need it to place the measurement block. 
	\draw [->] (controller) -- node[name=u] {$u$} (system);
	\node [output, right of=system] (output) {};
	
	% Once the nodes are placed, connecting them is easy. 
	\draw [draw,->] (input) -- node {$r$} (controller);
	\draw [->] (system) -- node [name=y] {$y$}(output); 
	\end{tikzpicture}
	\caption{Malha aberta de controle\label{mabertatikz}}
\end{figure}

\paragraph{} Uma forma de se compensar as perturbações do ambiente é realimentando a saída na entrada, calculando a diferença entre a referência e o valor medido. Assim, um valor de erro $e = e(t)$ é obtido e o sistema calcula o sinal $u$ conforme o erro evolui. A Figura \ref{mfechadatikz} mostra um esquema básico deste sistema.

\paragraph{} O controle malha aberta foi anteriormente verificado na referência \cite{redytton}.
\begin{figure}[!ht]
\centering
% The block diagram code is probably more verbose than necessary
\begin{tikzpicture}[auto, node distance=2cm,>=latex']
% We start by placing the blocks
\node [input, name=input] {};
\node [sum, right of=input] (sum) {};
\node [block, right of=sum] (controller) {Controlador};
\node [block, right of=controller, pin={[pinstyle]above:Perturbações},
node distance=3cm] (system) {Planta};
% We draw an edge between the controller and system block to 
% calculate the coordinate u. We need it to place the measurement block. 
\draw [->] (controller) -- node[name=u] {$u$} (system);
\node [output, right of=system] (output) {};
\node [block, below of=u] (measurements) {Medição};

% Once the nodes are placed, connecting them is easy. 
\draw [draw,->] (input) -- node {$r$} (sum);
\draw [->] (sum) -- node {$e$} (controller);
\draw [->] (system) -- node [name=y] {$y$}(output);
\draw [->] (y) |- (measurements);
\draw [->] (measurements) -| node[pos=0.99] {$-$} 
node [near end] {$y_m$} (sum);
\end{tikzpicture}
\caption{Malha fechada de controle\label{mfechadatikz}}
\end{figure}

\section{Bancada}
A bancada da ponte rolante presente no Laboratório de Automação e Controle está esquematizada na Figura \ref{bancadaEsquematico}. Serão detalhados os componentes desta bancada, de modo a se entender o papel de cada um deles. Observe que no esquemático falta a câmera, que é um sensor que fica de frente para a bancada. 

\begin{figure}[hbt]
\centering
  \includegraphics[width=0.8\textwidth]{figs/fundamentos/bancadaEsquematico}
  \caption{Esquemático da Bancada Utilizada para o Experimento \cite{redytton}\label{bancadaEsquematico}}
\end{figure}

\subsection{Controlador Lógico-Programável}
\begin{figure}[!ht]
  \centering
    \includegraphics[width=0.8\textwidth]{figs/fundamentos/CLP.jpg}
    \caption{CLP com identificação de elementos\label{CLPcomentado}}
\end{figure}

\paragraph{} O controlador lógico-programável (CLP - ver Figura \ref{CLPcomentado}) é a espinha dorsal da bancada. Ele é responsável por executar os comandos de controle sobre todos os elementos que estão conectados a ele. A programação do CLP é realizada via computador, mas, uma vez feito o \textit{download} do programa ao CLP, a execução ocorre independentemente do computador, contato que o CLP esteja em modo de execução.
\paragraph{} O CLP utilizado é fabricado pela \textit{Allen Bradley}, modelo Logix5560M03SE. Tal modelo possui memória lógica e de dados de 750 KiB, e memória de \textit{I/O} de 494 KiB. Há quatro módulos no \textit{chassis} do controlador:
\begin{itemize}
  \item O próprio controlador;
  \item \textit{SERCOS Interface};
  \item DeviceNET;
  \item EtherNet/IP.
\end{itemize}
\paragraph{} Cada um desses módulos será apresentado posteriormente com mais detalhes. Além dos módulos, o controlador ainda possui um \textit{switch} liga/desliga presente no \textit{chassis}. No módulo Logix, há uma chave responsável por alterar o modo de funcionamento do mesmo. As posições possíveis dessa chave são:
\begin{itemize}
  \item RUN;
  \item REM;
  \item PROG;
\end{itemize}
\paragraph{} Na prática, o modo REM se divide em dois modos: REM RUN e REM PROG. A maneira de se diferenciar os dois é observar, no módulo Logix, o estado do LED indicador de modo RUN quando a chave estiver na posição REM.
\paragraph{} No modo RUN, o controlador apenas roda o programa presente em sua memória; não há qualquer comunicação remota. No modo PROG, o controlador não roda nenhum programa; ele apenas pode receber um novo código. Nos modos REM, há a comunicação com o computador, permitindo verificar valores de variáveis de interesse e alterar, se necessário, o programa a ser rodado pelo controlador. O programa presente no controlador, em modo REM, só roda o código se estiver no modo REM RUN; se for necessário atualizar o programa, o modo deve ser o REM PROG.

\subsection{SERCOS Interface}

Sercos é um barramento digital de automação que interconecta controladores, \textit{drives}, dispositivos de entrada/saída e atuadores para máquinas e sistemas controlados numericamente. Foi projetado para comunicação serial de alta velocidade de dados em sistemas de tempo real por meio de fibra ótica (Sercos I \& II) ou um cabo Ethernet Industrial (Sercos III). Sercos é um padrão internacional \cite{sercos}.

Num sistema Sercos, todas as malhas que contém servomotores são normalmente fechadas no \textit{drive}. Isto reduz a carga computacional no CLP, permitindo-o sincronizar mais eixos do que conseguiria caso contrário. Além disso, fechar a malha dos servomotores com o \textit{driver} ajuda a reduzir o efeito do atraso de transporte entre o controle de movimento e o \textit{driver} \cite{sercos}.

Nesta bancada, o CLP deve se comunicar com o servomotor (MPL-A310F-SJ22AA) através do \textit{drive} Kinetix (2094-AC05MP5), de forma a movimentar o carrinho segundo uma trajetória planejada ou segundo uma lei de controle em malha fechada executando no CLP. Essa comunicação se dá por um par de fibras óticas full-duplex, conforme se observa na Figura \ref{CLPcomentado}, o que caracteriza uma rede Sercos I ou II.

\subsection{Line Interface Module 2094-AL09}

Este módulo não apareceu no esquemático da Figura \ref{bancadaEsquematico}, mas ele tem a função essencial de interfacear a rede trifásica com o servo \textit{drive}, permitindo o acionamento do motor. Na Figura \ref{LineInterfaceModule}, se observa que há três conjuntos de disjuntores nesse módulo: CB1 - liga ou desliga a rede trifásica do \textit{drive}  -, CB2 - fornece tensão monofásica ao servo \textit{drive} - e CB3 - liga as duas fontes de tensão DC de 24V, responsáveis pelas entradas e saídas digitais do módulo e alimentação do freio do motor \cite{redytton}.

\begin{figure}[!ht]
  \centering
    \includegraphics[width=0.7\textwidth]{figs/fundamentos/LineInterfaceModule}
    \caption{Line Interface Module modelo 2094-AL09 da Allen Bradley \cite{redytton}\label{LineInterfaceModule}}
\end{figure}

\subsection{Drive Kinetix 6000 da Allen Bradley}

Por meio da Interface Sercos, o CLP comanda o drive que então fornece potência ao motor. O drive controla o servomotor por meio de pulsos PWM. A Figura \ref{kinetix6000} apresenta um Drive Kinetix 6000 da Allen Bradley similar ao utilizado no laboratório. Mais detalhes sobre o funcionamento do drive estão disponíveis em \cite{redytton} e \cite{kinetix6000usermanual}.

\begin{figure}[!ht]
  \centering
    \includegraphics[width=0.3\textwidth]{figs/fundamentos/kinetix6000.jpg}
    \caption{Drive Kinetix 6000 da Allen Bradley\label{kinetix6000}}
\end{figure}

\subsection{Servomotor}

Um servomotor é um atuador rotatório que permite controle preciso da posição angular. O motor consiste de um motor acoplado a um sensor para a realimentação de posição/velocidade. Também é necessário um drive servo para completar o sistema. O \textit{drive} usa o sensor de realimentação para controlar precisamente a posição angular do motor, ou seja, é uma operação em malha fechada. Assim, usando servomotores em malha fechada, tem-se uma alternativa de alto desempenho aos motores de passo e de indução \cite{defServoMotores}.

O servomotor MPL-A310F-SJ22AA da Allen-Bradley foi utilizado e está representado na Figura \ref{servomotor}. O mesmo é composto por um motor indutivo de tensão nominal $230V_{\mathrm{ac}}$ e um encoder do tipo StegmanHiperface, que mede posição de forma absoluta e velocidade de forma incremental \cite{redytton}.

\begin{figure}[!ht]
  \centering
    \includegraphics[width=0.3\textwidth]{figs/fundamentos/servomotor}
    \caption{Servomotor modelo MPL-A310F-SJ22AA\label{servomotor}}
\end{figure}

\subsection{Sensores indutivos}
Os sensores indutivos são elementos detectores de presença, particularmente de objetos metálicos. Eles funcionam através da variação de campo magnético ocasionada pela presença do objeto a ser identificado. Tal variação de campo magnético provoca uma variação de corrente dentro do sensor, alterando seu estado.

Na presente bancada, há 6 sensores indutivos da família 871TM, similares ao da Figura \ref{sensorIndutivo}, fabricados pela \textit{Allen-Bradley}. Eles são alimentados com tensão de 24 V, que está dentro dos limites padrão. São sensores feitos de aço, adaptados a ambientes industriais.

\begin{figure}[!ht]
  \centering
    \includegraphics[width=0.3\textwidth]{figs/fundamentos/sensorIndutivo}
    \caption{Sensor indutivo 871T-R8B18 \cite{redytton}\label{sensorIndutivo}}
\end{figure}

A importância desses sensores é enorme. O motivo é que a câmera às vezes falha para obter a posição atual e tem um limite de frequência devido ao processamento interno que ela tem de realizar. Com os sensores indutivos, tem-se um processamento rápido para identificar se o motor está na região próxima dos limites. Assim, a rotina de segurança que trava o motor depende diretamente desses sensores indutivos.

\subsection{Câmera PresencePlus}

\section{Redes Utilizadas}
\subsection{DeviceNET}
\paragraph{}DeviceNET é um protocolo de baixo nível da camada de aplicação, voltado a ambientes industriais. É responsável pela interconexão de dispositivos visando ao compartilhamento de dados. Foi desenvolvido pela \textit{Allen-Bradley}, em cima da tecnologia CAN (\textit{Control Area Networking}) desenvolvida pela \textit{Bosch}. Tal rede suporta comunicação entre dispositivos de baixo nível, como sensores e atuadores, e dispositivos de alto nível, como o computador e o CLP. 
\paragraph{}O DeviceNET é uma combinação entre a camada física disponibilizada pelo CAN e um protocolo industrial, o CIP (\textit{Common Industrial Protocol}), que rege as redes industriais em geral. Permite uma rápida configuração entre dispositivos a \textit{byte}, suportanto tanto dispositivos analógicos quanto digitais. Permite velocidades de transmissão de até 500 kbps, sendo bem mais lenta que uma rede Ethernet.
\paragraph{}No experimento, a rede DeviceNET é utilizada para a conexão entre o CLP e os sensores indutivos presentes na planta. Devido à flexibilidade dos sensores e da rede, suas informações podem ser trabalhadas tanto no modo analógico quanto no modo digital. Para fins de detecção de fim-de-curso, entretanto, o modo utilizado é o digital, uma vez que não será preciso determinar a distância entre sensor e carrinho; apenas é necessário verificar se o carrinho está na área de detecção do sensor. O módulo responsável pelo gerenciamento da rede DeviceNET é o 1756-DNB.

\begin{comment}
https://en.wikipedia.org/wiki/DeviceNet
http://ab.rockwellautomation.com/Networks-and-Communications/DeviceNet-Network
http://www.rtaautomation.com/technologies/devicenet/
https://www.odva.org/Portals/0/Library/Publications_Numbered/PUB00122R1_CIP_Brochure_ENGLISH.pdf
\end{comment}

\subsection{Ethernet-IP}
\paragraph{}A rede Ethernet-IP, desenvolvida em meados dos anos 1990, é um tipo de rede Ethernet voltada ao ambiente industrial, seguindo o CIP, assim como o DeviceNET. É uma rede robusta, organizada segundo o modelo OSI de 7 camadas, que permite conexão com dispositivos conectados a redes Ethernet padrão; permite a passagem de dados via pacotes TCP ou UDP; é indicada em aplicações que exigem uma transferência rápida de dados (principalmente, em aplicações de tempo real).
\paragraph{}O uso da rede Ethernet-IP é vantajoso no sentido de que permite a conexão entre vários nós ligados entre si, o que não é possível com o RS-232, por exemplo. Porém, devido ao próprio funcionamento da rede, fica mais difícil obter os dados, pois, ao contrário de uma entrada serial, é necessário lidar com toda uma estrutura baseada no TCP/IP, por exemplo. Além disso, como o Ethernet exige um tamanho mínimo de \textit{frame} para transmissão de dados de cerca de 64 \textit{bytes}, a eficiência da transmissão pode ser afetada.
\paragraph{}No presente experimento, a rede Ethernet-IP é utilizada para receber dados de inspeção da câmera, notadamente a posição da bola de isopor presa ao barbante. Além disso, ela é utilizada na transferência de programas entre o computador e o CLP (através do módulo Ethernet 1756-ENBT/A), uma vez que ela provê uma comunicação mais rápida do que o RS-232.

\begin{comment}
https://en.wikipedia.org/wiki/Industrial_Ethernet
https://en.wikipedia.org/wiki/EtherNet/IP
https://www.odva.org/Technology-Standards/EtherNet-IP/Overview
http://www.rtaautomation.com/technologies/ethernetip/
http://www.rockwellautomation.com/global/products-technologies/integrated-architecture/ethernet-ip.page
http://www.rtaautomation.com/technologies/ethernetip/
\end{comment}

\section{Programação do CLP}
\subsection{Visão geral}

\subsection{Linguagem \textit{ladder}}

\subsection{Texto Estruturado}


%Resultados
%TCIDATA{LaTeXparent=0,0,relatorio.tex}
\chapter{Resultados\label{chap:Resultados}}

\resumodocapitulo{Este capítulo apresenta resultados de simulação, os procedimentos experimentais realizados e seus resultados.}


\section{Câmera}

Os problemas com a câmera se resumiram à configuração da rede, à calibração e à programação. O mais difícil inicialmente foi a programação, mas se tornou bem simples depois.

\subsection{Configuração da Rede}
A câmera estava com um \textit{firmware} antigo e foi necessário obter o arquivo 2014R1B PresencePLUS Firmware\footnote{Atualização de Firmware da Câmera - \url{http://info.bannerengineering.com/_dav/cs/idcplg?IdcService=GET_FILE&RevisionSelectionMethod=LatestReleased&dDocName=B_4170042}. Acesso em 29/11/2015.} que é o programa de atualização da \textit{Banner Engineering} para esta câmera. Bastou executá-lo no computador, estando a câmera conectada ao \textit{switch} do laboratório assim como o computador estava, ambos por cabos Ethernet. Antes desta atualização, o \textit{software} da câmera não permitia selecionar a opção Ethernet/IP de forma a permitir utilizar o módulo Ethernet/IP do CLP.

Uma vez atualizado o \textit{firmware}, foi obtido o arquivo EDS\footnote{Arquivo EDS disponível em \url{http://www.bannerengineering.com/en-US/products/sub/78\#ui-tabs-37}. Acesso em 29/11/2015.} da câmera que foi então integrado ao \textit{software} da Rockwell no computador por meio do programa \textit{EDS Hardware Installation Tool} da própria Rockwell, parte do RSLinx.

Após os procedimentos anteriores, adiciona-se a câmera como um módulo genérico usando o \textit{software} RSLogix, conforme instruções da Banner Engineering \cite{presencePlusEthernetIP}. Foi também necessário utilizar o RSLinx para verificar se a câmera  estava conectada ao computador e o \textit{software} da PresencePlus para identificar a câmera pelo endereço IP, uma vez que a rede utilizada é Ethernet/IP. A Figura \ref{linxcamera} mostra a janela do RSLinx com a câmera reconhecida.

\begin{figure}[!ht]
\centering
\includegraphics[width=\linewidth]{figs/resultados/camera/cameralinx}
\caption{RSLinx com câmera reconhecida \label{linxcamera}}
\end{figure}

\subsection{Calibração}

A câmera permite fazer medidas de distâncias em \textit{pixels}. De forma a se converter essa distância para milímetros, uma barra de alumínio com marcas e tamanho conhecido é utilizada. É importante primeiro calibrar o sistema, para se saber se há deformação de pixels significante ao longo da distância de interesse, nomeadamente o tamanho da barra de alumínio sendo utilizada, cerca de $532$mm. No PresencePlus P4 GEO 1.3, um programa é feito, com imagem de referência conforme Figura \ref{cameracalibracao}, que usa várias ferramentas de detecção de borda (ferramenta \texttt{Locate}) para identificar as posições de cada uma das marcas pretas da barra. Seis marcas foram feitas e a Tabela \ref{relacoesmmpx} apresenta os resultados para cada seção. A distância entre duas marcas é de 10cm, com exceção da distância entre P0 e PEND que é o comprimento total da barra.

\begin{figure}[!ht]
\centering
\includegraphics[width=0.9\linewidth]{figs/resultados/camera/programa}
\caption{Programa PresencePLUS para calibração da câmera \label{cameracalibracao}}
\end{figure}

\begin{table}[!ht]
\centering
\caption{Relações mm/px para diferentes seções da barra de alumínio \label{relacoesmmpx}}
	\begin{tabular}{|c|c|c|c|}
	\hline
		Seção 1 & Seção 2 & Distância (px) & mm/px\\ \hline
		P0 & P10 & 160 & 0.625\\ \hline
		P10 & P20 & 173 & 0.578\\ \hline
		P20 & P30 & 176 & 0.568\\ \hline
		P30 & P40 & 173 & 0.578\\ \hline
		P40 & P50 & 163 & 0.613\\ \hline
		P0 & PEND & 893 & 0.596\\ \hline
	\end{tabular}
\end{table}

O maior desvio da quantidade de milímetros por \textit{pixel} das seções em relação à da barra inteira é de aproximadamente 4.93\%. Há algumas imprecisões na maneira como os traços foram desenhados e é possível que o erro seja menor.

\subsection{Programação}
A programação da câmera se inicia obtendo uma imagem de referência e adicionando-se ferramentas de detecção de pontos de interesse, como mostra a Figura \ref{cameracalibracao}. A ferramenta \texttt{Locate} é responsável por detectar as bordas da bolinha, e a ferramenta \texttt{Geometric} detecta o centro da mesma. Também é possível adicionar ferramentas que fazem operações matemáticas (ferramenta \texttt{Math}), executam medições (ferramenta \texttt{Measure}), assim como as que enviam dados pela rede (ferramenta \texttt{Communication}), o que é essencial para se comunicar com o CLP.

\section{Calibração do Servomotor\label{calibracaoServomotorSecao}}

O RSLogix tem o bloco \texttt{MAJ} \textendash{} \textit{Motion Axis Jog} \textendash{} que permite alterar a velocidade do motor enquanto ele se movimenta. No entanto, o bloco espera que a entrada tenha unidade $[\mathrm{u}/\mathrm{s}]$ (unidades por segundo) ao invés de alguma unidade no SI tal como $[\mathrm{mm}/\mathrm{s}]$. Devido a isso, foi necessária uma calibração do sistema. O procedimento era anotar a posição inicial $x_0$, definir um tempo $\Delta t$ no qual o carrinho se movimenta a uma velocidade $v$ em $[\mathrm{u}/\mathrm{s}]$ e, após o movimento,  registrar a posição final $x_f$. As posições inicial e final são medidas em milímetros. Com isso, calculava-se a velocidade em $[\mathrm{mm}/\mathrm{s}]$. Alguns ensaios são realizados seguindo esse procedimento e a média é considerada como o valor de uma unidade, resultando em aproximadamente $71.32\mathrm{mm}$. Os dados de calibração estão na Tabela \ref{calibracaoServomotor}.

\begin{table}[!ht]
\centering
\caption{Dados de calibração do servomotor, média obtida é de 71.32 mm/unidade\label{calibracaoServomotor}}
\begin{tabular}{|c|c|c|c|c|c|}
\hline
	$x_0$ - [mm] & $x_f$ - [mm] & $\Delta t$ - [s] & Velocidade - [u/s] & Velocidade - [mm/s] & mm/u\\ \hline
2 &	71.8  &	2   &	0.5 &	34.9   & 	69.8\\ \hline
6 & 76.1  &	2   &	0.5 &	35.05  &	70.1\\ \hline
6 &	188	  &  5   &	0.5	&   36.4   &	72.8\\ \hline
6 &	185   &	2.5 &	1	& 	71.6   &	71.6\\ \hline
6 &	77    &	10  &	0.1	&   7.1    &	71\\ \hline
6 &	296.5 &	20	&   0.2 & 	14.525 &	72.625\\ \hline\end{tabular}
\end{table}

\section{Modelo no Espaço de Estados}

A partir da teoria apresentada na Subseção \ref{reducaoModal}, foram desenvolvidas algumas rotinas em linguagem Julia \cite{julia} para obtenção das matrizes reduzidas. Na rotina, a ordem da matriz de saída é um parâmetro, mas para os propósitos deste projeto só foi testado o sistema reduzido a quatro estados. A Seção \ref{reducaoModalPrograma} dos anexos apresenta o código completo.

As matrizes obtidas para o modelo reduzido são dadas \begin{align}
\begin{array}{ll}
	\mathbf{A_R} &= \left[\begin{array}{cccc}
		-0.0881  & -3.8389 &         0 &         0\\
    3.8389 &   -0.0881 &         0 &         0\\
         0 &         0 &   -0.1061 &  -10.8145\\
         0 &        0 &   10.8145 &   -0.1061\\
	\end{array}\right],\\
	\mathbf{B_R} &= 10^3\left[0.3313,\;
    0.0066,\;
    1.3549,\;
    0.0163\right]^{\mathrm{T}},\\
	\mathbf{C_R} &= \left[-0.0003,\;0.0148,\;0.0001,\;-0.0029\right]\;\mathrm{e}\\
	D_R &= -0.0906,
\end{array} \label{modeloReduzidoSemEpsilon}
\end{align} sendo que é fácil observar que os autovalores da matriz $\mathbf{A_R}$ são dados por $-0.0881\pm 3.8389j$ e $-0.1061\pm 10.8145j$, valores extraídos dos blocos diagonais dessa matriz. Esse tipo de estrutura já era esperada quando se apresentou a técnica de redução modal na subseção \ref{reducaoModal}. O sistema composto por essas matrizes é dado por \begin{align}
	\begin{array}{ll}
		\mathbf{\dot{z}} &= \mathbf{A_R}\mathbf{z} + \mathbf{B_R}u(t)\;\mathrm{e}\\
		y &= \mathbf{C_R}\mathbf{z} + D_Ru(t).
	\end{array}\label{modeloEspacoDeEstadosSemAtraso}
\end{align}


No sistema da Equação \ref{modeloEspacoDeEstadosSemAtraso}, ainda não se compensou pela transferência direta diferente de zero. Note que ela era zero antes da redução, mas a perca do ganho estático dos outros modos do sistema levou a esse $D_R$ não nulo. Quando se analisa a resposta ao degrau em malha aberta para este caso, Figura \ref{modeloMalhaAberta}, nota-se que o atraso é cerca de 0.3s. Para ser preciso, $\epsilon = 0.313$s. Com esse valor de atraso, calculam-se novas matrizes $\mathrm{B}_D$ e $D_D$ para substituir $\mathrm{B}_R$ e $D_R$, respectivamente, conforme Equações \ref{novoBD} e \ref{novoDD}. Utilizando exatamente esse atraso de $0.313$s, a transferência direta se tornaria aproximadamente zero. No entanto, esses modelos no espaço contínuo serão discretizados com período $T_s = 0.1$s. Daí, o $\epsilon$ mais próximo realizável é 0.3s e seu uso resulta nas novas matrizes \begin{align}
\begin{array}{ll}
	\mathbf{B_D} &= 10^3\left[0.1255,\;
    0.2974,\;
   -1.3039,\;
   -0.1504\right]^{\mathrm{T}}\;\mathrm{e}\\
	D_D &= -0.0685,
\end{array} \label{modeloReduzidoComEpsilon}
\end{align} que então fazem parte do novo sistema reduzido dado por\begin{align}
	\begin{array}{ll}
		\mathbf{\dot{z}} &= \mathbf{A_R}\mathbf{z} + \mathbf{B_D}u(t-\epsilon)\;\mathrm{e}\\
		y &= \mathbf{C_R}\mathbf{z} + D_D u(t-\epsilon).
	\end{array}\label{modeloEEComAtraso}
\end{align}

\begin{figure}[!ht]
\centering
\includegraphics[width=0.8\linewidth]{figs/resultados/modelo/respostaMalhaAberta}
\caption{Resposta ao degrau para modelos reduzidos com atraso e sem atraso\label{modeloMalhaAberta}}
\end{figure} 

Apesar de $\epsilon=0.3$s ser próximo de 0.313s, esse atraso não reduz muito a transferência direta, como se pode observar analisando $D_D$, Equação \ref{modeloReduzidoComEpsilon}.



 O sistema em malha aberta oscila muito, mas é estável. Observa-se que a resposta decai conforme o tempo passa, conforme Figura \ref{modeloMalhaAberta25s}. O objetivo do controle é reduzir ao máximo essas oscilações, tendo uma trajetória o mais suave possível de um ponto inicial a um ponto final. 
\begin{figure}[!ht]
\centering

\includegraphics[width=0.8\linewidth]{figs/resultados/modelo/respostaMalhaAberta25s}
\caption{Resposta ao degrau para modelos reduzidos com atraso e sem atraso, 25 segundos\label{modeloMalhaAberta25s}}
\end{figure}

\subsection{Discretização}
 O controle será feito conforme a Seção \ref{controle}, mas primeiro deve-se discretizar o sistema.  A função \texttt{c2d} \cite{c2d} do MATLAB é utilizada, resultando nas matrizes \begin{align}
 \begin{array}{ll}
 	\mathbf{A} &= \left[\begin{array}{cccc}
	0.9191&   -0.3712&         0&         0\\
    0.3712&    0.9191&         0&         0\\
         0&         0&    0.4651&   -0.8733\\
         0&         0&    0.8733&    0.4651\\
 \end{array}\right],\\
 	\mathbf{B} &= \left[6.5818,\;
   31.2517,\;
  -98.5991,\;
  -75.6695\right]^{\mathrm{T}},\\
   \mathbf{C} &= \left[-0.0003,\;0.0148,\;0.0001,\;-0.0029\right],\\
   D &= -0.0685.
 \end{array}
 \end{align}
 
 Os polos de $\mathbf{A}$ são $0.9191 \pm 0.3712j$ e $
   0.4651 \pm 0.8733j$. Como esses polos estão dentro do círculo unitário, o sistema continua estável, apesar de manter as oscilações, pelo fato dos polos terem parte imaginária. Vale notar que as matrizes $\mathbf{C}$ e $D$ são iguais a $\mathbf{C_R}$ e $D_D$ do sistema contínuo, respectivamente, conforme Equações \ref{modeloReduzidoSemEpsilon} e \ref{modeloReduzidoComEpsilon}.
 
\subsection{Controle}

 A parte mais difícil do controle é a escolha dos polos. Algumas simulações foram realizadas e observava-se se o sistema era estável e se oscilava muito. O projeto final considerou 5 polos: $\left[0.6,\;0.6,\;0.6,\;0.5\pm 0.4j\right]$. O primeiro polo se deve ao integrador que foi adicionado ao sistema. Os polos com parte imaginaria foram escolhidos baseados nos polos originais do sistema. A ideia é evitar forçar muito esses polos para longe, evitando que os ganhos sejam altos e economizando em atuação. Os ganhos obtidos para esse caso são dados por \begin{align}
 \begin{array}{ll}
	 	\mathbf{K_p} &= \left[0.0080,\;0.0096,\;-0.0046,\;-0.0004\right],\\
 	K_i &= 0.1953.\\ 	
 \end{array}\label{ganhosObtidos}
 \end{align}
 
 O código para calcular os ganhos está disponível na Seção \ref{projetoControlador}.

\section{Simulação}

 Antes de se implementar o controle na planta, é essencial a realização de simulações. Elas são rápidas de serem realizadas e permitem averiguar se um projeto é estável ou não. As simulações aqui realizadas incluem o sistema em malha aberta com entrada tipo rampa e a entrada suave, utilizada por \cite{rafaelMestrado}. Depois da apresentação dos resultados em malha aberta, são apresentados os resultados em malha fechada sem considerar ruído. Por fim, os resultados com um ruído somado são apresentados.
 
 O resultado da simulação em malha aberta é apresentado nas Figuras \ref{respostaMalhaAbertaRampa} e \ref{respostaMalhaAbertaEntradaSuave}. Nota-se que a resposta à rampa teve diversas oscilações que demoram a atenuar-se. Já a entrada calculada por \cite{rafaelMestrado} teve uma ótima resposta. A desvantagem da implementação em malha aberta é que ela não resiste a perturbações e essas sempre ocorrem, ainda mais quando se considera o fundo do oceano.
 
 \begin{figure}[!htb]
    \centering
    \begin{minipage}{.45\textwidth}
        \centering
        
        \includegraphics[width=1\linewidth]{figs/resultados/simulacao/respostaMalhaAbertaRampa}
        \label{respostaMalhaAbertaRampa}
        \caption{Resposta do Sistema em Malha Aberta para Excursão de 30cm, entrada rampa}g
    \end{minipage}%
    \hspace{0.1cm}
    \begin{minipage}{0.45\textwidth}
        \centering
        
        \includegraphics[width=1\linewidth]{figs/resultados/simulacao/respostaMalhaAbertaRefTopo}
        \label{respostaMalhaAbertaEntradaSuave}
        \caption{Resposta do Sistema em Malha Aberta para Excursão de 30cm, entrada suave calculada}
    \end{minipage}
\end{figure}

 A solução para lidar com perturbações é fechar a malha. Para isso, uma vez obtidos os ganhos, Equação \ref{ganhosObtidos}, fez-se o projeto do sistema em malha fechada em Simulink \cite{simulink}. A Figura \ref{topModel} apresenta os principais blocos e suas conexões. Nos somadores, as referências para a posição de topo, que é a posição do carrinho, e para a posição de fundo, que é a posição da bolinha, são as entradas do sistema. O bloco \texttt{PLANTA} é o modelo sem redução modal. No caso deste projeto, a matriz do sistema sem redução de ordem é de dimensão $400\times 400$. Os blocos \texttt{Signal Builder} representam um sinal tipo rampa definido para comparação com as trajetórias suaves de topo e fundo \cite{rafaelMestrado}. 
 

\begin{figure}[!ht]
\centering

\includegraphics[width=0.9\linewidth]{figs/resultados/simulink/top}
\caption{Esquema principal para simulação em Simulink\label{topModel}}
\end{figure}

 O bloco atraso, Figura \ref{blocoAtraso}, tem como saída o valor predito pelo modelo reduzido menos esse mesmo sinal atrasado em 3 períodos de amostragem, uma vez que $\epsilon \simeq 3T_s$. Esse valor é somado à diferença entre a referência de fundo e a saída da planta para então ser a entrada do bloco de controle. O bloco \texttt{CONTROLE\_FILTRO} é detalhado na Figura \ref{blocoControle} e utiliza os ganhos da Equação \ref{ganhosObtidos}. O filtro de Kalman utilizou, como matrizes de covariância, $Q = 0.01^2 I_4$ e $R = 0.4^2$; tais valores foram obtidos de forma empírica.

\begin{figure}[!ht]
\centering

\includegraphics[width=0.5\linewidth]{figs/resultados/simulink/atraso}
\caption{Bloco de Atraso -- saída é o valor antecipado menos um valor antigo\label{blocoAtraso}}
\end{figure}

\begin{figure}[!ht]
\centering

\includegraphics[width=0.6\linewidth]{figs/resultados/simulink/modeloReduzido}
\caption{Modelo reduzido -- não utiliza atraso, utilizado para predizer a saída\label{blocoModeloReduzido}}
\end{figure}

\begin{figure}[!ht]
\centering
\includegraphics[width=0.9\linewidth]{figs/resultados/simulink/controle}
\caption{Bloco de controle com filtro de Kalman\label{blocoControle}}

\end{figure}

 O resultado das simulações sem ruído é apresentado nas Figuras \ref{respostaMalhaFechadaRampa} e \ref{respostaMalhaFechadaRefTopoFundo}. Nota-se uma ótima melhoria na resposta à rampa, enquanto a resposta à entrada suave se mantém boa.


\begin{figure}[!htb]
    \centering
    \begin{minipage}{.45\textwidth}
        \centering
       
        \includegraphics[width=1\linewidth]{figs/resultados/simulacao/respostaMalhaFechadaRampa}
        \label{respostaMalhaFechadaRampa}
         \caption{Resposta do Sistema em Malha Fechada para Excursão de 30cm, entrada rampa}
    \end{minipage}%
    \hspace{0.1cm}
    \begin{minipage}{0.45\textwidth}
        \centering
      
        \includegraphics[width=1\linewidth]{figs/resultados/simulacao/respostaMalhaFechadaRefTopoFundo}
        \label{respostaMalhaFechadaRefTopoFundo}
        \caption{Resposta do Sistema em Malha Fechada para Excursão de 30cm, entrada suave calculada para topo e fundo}
    \end{minipage}
\end{figure}

Algo essencial é analisar a resistência ao ruído. Ela depende não só da malha fechada, mas também do filtro de Kalman que foi implementado. Os parâmetros $Q$ e $R$ escolhidos anteriormente são essenciais para uma boa atenuação de ruídos. O ruído foi inserido na simulação conforme a Figura \ref{simulacaoComRuidoSimulink}. O que se nota é que o sinal de entrada do bloco de controle sempre terá algum erro, o que seria um erro no sistema de medição. O resultado da simulação é apresentado na Figura \ref{respostaMalhaAbertaRefTopoRuido}. É importante observar a diferença entre o sinal de referência antes e depois de ser somado o ruído. Conforme a Figura \ref{entradaControladorYERR} apresenta, esse sinal tem grandes variações comparado ao original; mesmo assim, o resultado final está bem controlado, evidenciando a importância do controle em malha fechada com a presença do Filtro de Kalman como observador. 

\begin{figure}[!ht]
\centering

\includegraphics[width=0.8\linewidth]{figs/resultados/simulacao/simulacaoComRuido}
\caption{Sistema com um ruído branco adicionado\label{simulacaoComRuidoSimulink}}
\end{figure}


\begin{figure}[!htb]
    \centering
    \begin{minipage}{.45\textwidth}
        \centering
        
        \includegraphics[width=1\linewidth]{figs/resultados/simulacao/respostaMalhaAbertaRefTopoRuido}
        \label{respostaMalhaAbertaRefTopoRuido}
        \caption{Resposta do Sistema em Malha Fechada para Excursão de 30cm, entrada suave, com ruído}
    \end{minipage}%
    \hspace{0.1cm}
    \begin{minipage}{0.45\textwidth}
        \centering
               \includegraphics[width=1\linewidth]{figs/resultados/simulacao/entradaControladorYERR}
        \label{entradaControladorYERR}
        \caption{Entrada do controlador antes e depois do ruído para cada instante}

    \end{minipage}
\end{figure}


\section{Resultados Experimentais}
\subsection{Considerações Iniciais}
De posse das simulações feitas, assumindo-se que o sistema se comporta de forma estável e desejada com os polos escolhidos e as matrizes de covariância obtidas para o Filtro de Kalman, prosseguiu-se com a validação experimental do controle proposto. Para tal, em conjunção com os programas necessários desenvolvidos no RSLogix, foram desenvolvidos módulos e códigos em linguagem Python para implementar o controle. As três principais razões de se utilizar Python são:
\begin{enumerate}
\item A linguagem Python, conforme discutido na Seção \ref{opcSubSection}, possui suporte ao módulo OpenOPC \cite{OpenOPC}, necessário para se efetuar a troca de informações entre o preditor de Smith e a planta por intermédio do CLP;
\item Embora haja outras linguagens e \textit{softwares} que suportem OPC, como o próprio MATLAB, além de existirem linguagens mais rápidas e ao mesmo tempo ricas em ferramentas como o C++, o uso de Python se deve ao fato de já ter o módulo OPC e ser razoavelmente rápido, talvez não tanto quanto o C++. Porém, dada a facilidade de se programar com o OpenOPC e os tempos de comunicação e execução dos programas feitos não interferirem com o tempo de amostragem, Python se torna uma opção extremamente viável para este projeto;
\item Além do módulo OpenOPC, facilidade de programação e tempos razoáveis, Python possui também suporte à orientação a objetos. Para programar estruturas como o preditor de Smith, tal abordagem é muito importante, visto que cada componente pode ser tratado como um objeto, facilitando a implementação. Além disso, a estrutura de programação é tal que outras formas de controle podem ser facilmente adaptadas à ponte rolante por meio da interface com Python; as alterações no programa do RSLogix são mínimas, caso sejam necessárias.
\end{enumerate}

Assim como nas simulações, foram considerados três experimentos a serem validados: trajetória em malha aberta; trajetória em malha fechada, considerando topo e fundo como rampas; e a trajetória considerada por Rafael \cite{rafaelMestrado}. Todas as trajetórias consideram excursão de 30 centímetros.

\subsection{Testes com Rampa - Malha Aberta e Malha Fechada}
O primeiro teste feito foi considerando uma trajetória em formato rampa, realizando um deslocamento de 30 centímetros em cerca de 2.5 segundos. Tal teste foi feito em malha aberta\footnote{Teste Experimental com Rampa em Malha Aberta --- \url{https://youtu.be/chrez0QEucU}. Acesso em 30/06/2016.} e o resultado se encontra na Figura \ref{malhaAbertaRampa}. 

O segundo teste realizado considerou a mesma trajetória do primeiro teste como referência tanto para o topo quanto para o fundo para um experimento com a malha fechada, utilizando-se o preditor de Smith\footnote{Teste Experimental com Rampa utilizando Preditor de Smith --- \url{https://youtu.be/QOuxlsT3gBA}. Acesso em 30/06/2016.}. O resultado se encontra na Figura \ref{malhaFechadaRampa}.

\begin{figure}[!ht]
\centering
\begin{minipage}{0.45\textwidth}
\centering
\includegraphics[width=1\linewidth]{figs/resultados/experimento/open_loop_ramp}
\caption{Resultado experimental para a trajetória rampa em malha aberta de 30 cm. \label{malhaAbertaRampa}}
\end{minipage}
\hspace{0.1cm}
\begin{minipage}{0.45\textwidth}
\centering
\includegraphics[width=1\linewidth]{figs/resultados/experimento/closed_loop_trajetoria_rampa}
\caption{Resultado experimental para a trajetória rampa em malha fechada de 30 cm. \label{malhaFechadaRampa}}
\end{minipage}
\end{figure}

Nota-se, pelos resultados apresentados, que cada um dos testes realizados com a rampa possuem vantagens e desvantagens entre si: a trajetória em malha aberta apresentou um sobressinal ligeiramente menor e melhor aproximação com o resultado final; entretanto, a trajetória em malha fechada foi melhor seguida durante o movimento. Em termos práticos, considerando a presente trajetória, o preditor de Smith realizou melhor a tarefa do acompanhamento da trajetória, mas o controle em malha fechada deixou a desejar, já que as oscilações não foram controladas. A Figura \ref{rampaComparativo} mostra os dois resultados comparados em um mesmo gráfico, subtraindo-se a posição inicial de cada teste.

\begin{figure}[!ht]
\centering
\includegraphics[width=.6\linewidth]{figs/resultados/experimento/rampa_comp}
\caption{Gráfico comparativo das respostas em malha aberta e malha fechada com trajetória rampa. \label{rampaComparativo}}
\end{figure} 

Motivos para o resultado ruim, ainda mais se comparado com o experimental na Figura \ref{malhaFechadaRampa}, incluem erros de calibração -- a câmera é desconfigurada rapidamente no laboratório de automação -- e o fato de termos utilizado OPC adicionava um atraso na comunicação para fechar a malha e às vezes o próximo valor medido não era lido rápido o suficiente para ser considerado no próximo passo de simulação. O filtro de Kalman possivelmente pode ter parâmetros melhores escolhidos com mais alguns testes experimentais, o que poderia também melhorar problemas com ruído.

\subsection{Teste com Trajetória Proposta - Malha Fechada}
O último teste realizado com o preditor de Smith envolve a trajetória utilizada por Rafael \cite{rafaelMestrado}. Nesse caso, a trajetória de topo e de fundo são ligeiramente diferentes, sugerindo uma trajetória próxima do esperado para um \textit{riser}. O resultado do experimento\footnote{Teste Experimental com Trajetória Proposta por Rafael\cite{rafaelMestrado} utilizando Preditor de Smith --- \url{https://youtu.be/lpPZ7HOG7hM}. Acesso em 30/06/2016.} se encontra na Figura \ref{experimentoRafael}. Nota-se que a saída é bem comportada, seguindo a trajetória; porém, o resultado final apresenta pequenas oscilações, além de um erro de medição. A Figura \ref{experimentoRafaelDetalhe} mostra em detalhe a saída, durante o tempo em que a trajetória é executada. 

\begin{figure}[!ht]
\centering
\begin{minipage}{.45\textwidth}
\centering
\includegraphics[width=1\textwidth]{figs/resultados/experimento/closed_loop_trajetoria_rafael}
\caption{Teste com a trajetória sugerida por Rafael utilizando o preditor de Smith. \label{experimentoRafael}}
\end{minipage}
\hspace{0.1cm}
\begin{minipage}{.45\textwidth}
\centering
\includegraphics[width=1\textwidth]{figs/resultados/experimento/closed_loop_trajetoria_rafael_detalhe}
\caption{Teste com a trajetória sugerida por Rafael, detalhando a execução da trajetória. \label{experimentoRafaelDetalhe}}
\end{minipage}
\end{figure} 

Deve-se notar que houveram alguns resultados indesejáveis nos testes apresentados, em particular os que envolvem o preditor. Em primeiro lugar, houve a presença de um sobressinal considerável nos testes com rampa -- cerca de $12.5\%$. Esse tipo de sobressinal pode ser extremamente prejudicial para a operação do \textit{riser}. Além do problema do sobressinal, houve o fato de que a saída oscila e apresenta um erro estático nos testes com o preditor, representados nas Figuras \ref{malhaFechadaRampa} e \ref{experimentoRafael}; embora a presença desses fatores faz o resultado experimental destoar dos resultados das simulações, o resultado geral foi satisfatório. Um resultado marcante foi a atenuação da saída proporcionado pelo uso do Filtro de Kalman, considerando as matrizes de covariância escolhidas; o fato da matriz $\mathbf{R}$ possuir valores maiores do que os elementos de $\mathbf{Q}$ sugere que foi priorizada a correção sobre a medida da posição de fundo.

%Conclusões
%TCIDATA{LaTeXparent=0,0,relatorio.tex}



\chapter{Conclusões}

\label{CapConclusoes}

Concluir


\section{Perspectivas Futuras}

Perspectivas futuras



%Bibliografia

\renewcommand{\bibname}{REFERÊNCIAS BIBLIOGRÁFICAS} 
\addcontentsline{toc}{chapter}{REFERÊNCIAS BIBLIOGRÁFICAS} 

\bibliographystyle{abnt-num}
\bibliography{relatorio}

%
\anexos 
\makeatletter 
%% não retirar estes comandos 
\renewcommand{\@makechapterhead}[1]{%
  {\parindent \z@ \raggedleft \setfontarial\bfseries          
\LARGE \thechapter. \space\space      
\uppercase{#1}\par     
\vskip 40\p@   
} 
} 
\makeatother
%
%%Anexo I: Descriçao do CD
%
%%TCIDATA{LaTeXparent=0,0,relatorio.tex}



\chapter{Descri��o do conte�do do CD}

\label{AnCD} 

Descrever CD.

%
%\refstepcounter{noAnexo}
%
% %Anexo II: Programas Utilizados


%TCIDATA{LaTeXparent=0,0,relatorio.tex}



\chapter{Programas utilizados}

\section{Redução modal}
\label {reducaoModalPrograma}
\lstinputlisting[language=Julia]{codes/ModalReduction.jl}

\section{Texto Estruturado}
\label {stsection}

\subsection{Inicialização do primeiro teste de malha aberta}
\label {stMAinit1}
\lstinputlisting{codes/init1.st}

\subsection{Inicialização do segundo teste de malha aberta}
\label {stMAinit2}
\lstinputlisting{codes/init2.st}

\subsection{Inicialização dos testes de malha fechada}
\label {stMFinit}
\lstinputlisting{codes/initP.st}

\subsection{Inicialização da rede DeviceNET}
\label{dninitST}
\lstinputlisting{codes/InitDNetST.st}

\subsection{Programa de controle em malha aberta}
\label{maprogST}
\lstinputlisting{codes/malhaAbertaT2.st}

\subsection{Programa de controle proporcional - malha fechada}
\label{mfprogST}
\lstinputlisting{codes/malhaFechadaP.st}

\section{Linguagem \textit{ladder}}
\subsection{Execução de \textit{trigger} da câmera}
\label{laddertrigger}
\begin{figure}[!ht]
\centering
\includegraphics[width=\linewidth]{figs/ladder/camera_trigger}
\caption{\textit{Trigger} da câmera}
\end{figure}

\subsection{Rotina de parada de emergência}
\label{emergencyladder}
\begin{figure}[!ht]
\centering
\includegraphics[width=\linewidth]{figs/ladder/parada}
\caption{Parada de emergência}
\end{figure}

\section{Programas da câmera}
\subsection{Detecção da posição horizontal da bolinha}
\label{ballhorzpos}
\begin{figure}[!ht]
\centering
\includegraphics[width=\linewidth]{figs/presence/programaCaptura}
\caption{Detecção da posição horizontal da bolinha}
\end{figure}
\refstepcounter{noAnexo}

%%Acrescente mais anexos conforme julgar necessário.

\end{document}
