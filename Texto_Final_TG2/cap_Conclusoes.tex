%TCIDATA{LaTeXparent=0,0,relatorio.tex}

\chapter{Conclusões}

\label{CapConclusoes}

O projeto desenvolvido até aqui é uma continuação do que foi desenvolvido por Rédytton \cite{redytton}; porém, muitos avanços foram realizados. Os avanços e melhorias realizados foram:
\begin{itemize}
\item calibração da câmera, de forma a se pegar os dados da trajetória de fundo, representada pela bolinha;
\item cálculo do fator de conversão entre a unidade de velocidade reconhecida pelo RSLogix e uma unidade de velocidade derivada do SI --- milímetros por segundo;
\item adoção do texto estruturado para se desenvolver as estratégias de controle a serem realizadas pelo CLP, uma vez que o texto é facilmente modificável para novas trajetórias, o que não ocorre com linguagens gráficas como o \textit{ladder};
\item modificação do protocolo de comunicação do CLP com o computador. O novo protocolo, Ethernet/IP, é consideravelmente mais rápido que uma comunicação baseada em RS-232, o que resulta em um tempo menor entre um teste e outro;
\item uso do OPC para comunicação entre computador e CLP. Dessa forma, uma estrutura complexa como o preditor de Smith pode ser programada em uma linguagem mais rica em termos de ferramentas do que o texto estruturado --- no caso, foi utilizada a linguagem Python com o módulo OpenOPC \cite{OpenOPC}.
\end{itemize}

No início do trabalho, não havia nenhum conhecimento prévio dos autores sobre programação em tempo real utilizando CLPs; os conhecimentos de programação CLP em geral eram bastante limitados. Além disso, nenhum conhecimento prévio da bancada existia também. No estágio atual, pode-se dizer que há relevante facilidade em se trabalhar com a bancada, apesar de não se ter entrado em detalhes de como funciona o servomotor, \textit{drive} e outros componentes.

O foco do projeto era desenvolver uma estrutura de controle não clássico para lidar com um sistema de ordem infinita, representado pelo conjunto carrinho, bolinha e barbante. Dentro dessa abordagem, procurou-se validar o experimento, por meio de um extenso tratamento matemático, de forma que ele representasse confiavelmente uma operação de re-entrada com um \textit{riser} real. O desenvolvimento matemático e simulações demonstraram a validade da abordagem em teoria. O sistema reduzido em espaço de estados de ordem baixa permitiu desenvolver uma estratégia de controle simples, como a realimentação de estados com canal integral e permitiu integrar de forma simples um preditor de Smith com Filtro de Kalman e controlar a planta de tal forma que a trajetória de fundo real acompanhasse sua referência com erro mínimo.

Embora os resultados experimentais não terem sido ideais, pode-se notar que o uso do preditor de Smith é um grande avanço em relação ao teste com malha aberta no que concerne ao acompanhamento da trajetória; porém, considerando todas as fontes de erro oferecidas pela planta em questão --- posicionamento da câmera, perdas de comunicação entre câmera, CLP e computador, atrito na esteira do carrinho, conversão de velocidades, atraso devido à comunicação (\textit{overhead}), entre outros fatores ---, nota-se que os resultados foram satisfatórios; em outras palavras, pode-se dizer que é válido o uso do preditor de Smith como estratégia de controle para uma operação de re-entrada de \textit{risers}, tendo como base os resultados encontrados.

\section{Problemas}
É importante indicar os problemas que ocorreram nessa trabalho. Na Figura \ref{rampaComparativo}, nota-se que o sistema em malha fechada aparenta ter atingido regime, mas demonstra um certo erro estático em relação a referência. Devido ao integrador no caminho direto, conforme a Figura \ref{mfechadatikzEspacoDeEstados}, esse sistema deveria ter erro em regime nulo, conforme \cite{OgataDiscrete:1995}. O motivo do erro ainda é desconhecido.

O uso da linguagem Python com OpenOPC pode ter sido uma causa de problemas. A leitura de posição ocorria no limite do tempo de amostragem, ou até demorava um pouco mais. O uso de texto estruturado, evitando transmissão de dados para o controle, poderia ter evitado isso. Recentemente, foi descoberto um exemplo em MATLAB que gera texto estruturado \cite{stGeneration}. Explorando essa ferramenta poderia facilitar a programação do controle, utilizando-se MATLAB, e ao mesmo tempo evitaria-se usar o computador para fechar a malha.

O filtro de Kalman teve parâmetros escolhidos empiricamente por meio de tentativa e erro. O ideal seria fazer-se medidas da imprecisão das variáveis de interesse na bancada, de alguma maneira, e então utilizar esses dados no filtro de Kalman, o que poderia fornecer resultados muito melhores.

Houve problemas para a alocação dos polos. Os ganhos obtidos na Equação \ref{ganhosObtidos} consideraram 3 polos em $z=0.6$ e um par complexo conjugado em $z=0.5\pm 0.4j$ e, na realidade, uma variação para polos todos iguais em, por exemplo, $z=0.4$, $z=0.1$ ou $z=0.8$, mesmo estando dentro do círculo unitário, resultavam em um sistema que divergia. Por que, mesmo esses polos sendo estáveis, o sistema não era estável? Em algumas simulações, notou-se que as constantes do filtro de Kalman influenciavam na estabilidade do sistema.

Um erro foi descoberto logo antes desse trabalho ser apresentado, que foi um sinal errado no cálculo do ganho estático. Especificamente, a Equação \ref{transferenciaDiretaFabricio} para a transferência direta do sistema reduzido tem o sinal errado e deveria ser dada por \begin{align}
	\mathbf{D_R} &= \mathbf{C_R}\mathbf{A_R}^{-1}\mathbf{B_R} - \mathbf{C}\mathbf{A}^{-1}\mathbf{B},
\end{align} conforme apresentado no trabalho de Fortaleza \cite{teseEugenio}. Nossas referências principais, Fabrício \cite{fabricioIFAC} e Rafael \cite{rafaelMestrado}, também apresentaram esse erro no sinal, provavelmente oriundo de um erro de digitação. Talvez esse cálculo errado da transferência direta esteja diretamente relacionado com o problema mencionado de polos estáveis causarem o sistema a divergir.

\section{Perspectivas Futuras}
Há muito a ser feito ainda. Além do foco em reduzir o erro em muitos dispositivos da planta, particularmente na câmera, há outros pontos de melhoria no projeto, como por exemplo:
\begin{itemize}
\item Utilizar uma melhor estratégia de calibração da câmera --- a atual não permite excursões maiores que 50 cm;
\item Estudar outras formas de controle para o \textit{riser}, e verificar o desempenho das mesmas em relação ao preditor de Smith;
\item Procurar melhorar a programação, otimizando os algoritmos desenvolvidos e utilizando ferramentas para acelerar os programas já feitos (por exemplo, o uso de \textit{threads}, já feito neste projeto). A razão de se acelerar os programas que fazem a comunicação por meio do OPC é impedir que seus custos de execução e \textit{overheads} interfiram no tempo de amostragem, comprometendo os resultados;
\item Fazer testes mais minuciosos com as estruturas já desenvolvidas, e utilizar Filtros de Kalman diferentes (por exemplo, filtros adaptativos). O objetivo aqui é atenuar ainda mais as perturbações que venham a ocorrer, visto que um \textit{riser} real é constantemente atingido por forças externas ao controle, como correntes maritmas e eventuais colisões com outros corpos.
\end{itemize}